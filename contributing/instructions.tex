\documentclass[11pt,fleqn]{article}

%% This first part is the document header, which you don't need to edit.
%% Scroll down to \begin{document}

\usepackage[latin1]{inputenc}
\usepackage{enumerate}
\usepackage[hang,flushmargin]{footmisc}
\usepackage{mdframed}
\usepackage{minted}
\usepackage{color}
\usepackage{datetime}
\usepackage{url}

\setlength{\oddsidemargin}{0px}
\setlength{\textwidth}{460px}
\setlength{\voffset}{-1.5cm}
\setlength{\textheight}{20cm}
\setlength{\parindent}{0px}
\setlength{\parskip}{10pt}

\newcommand{\mil}[2][java]{\mintinline{#1}|#2|}

\begin{document}
\title{Contributing}%Insert Title here
\author{Tim Magoun and Aravind Koneru}
\date{\textit{Compiled on} \today \hspace{1mm} at \currenttime}
\maketitle

\section*{Standards}
While we welcome contributions to our project, we want to ensure that all contributions are of high
quality and adhere to the standards set forth by this project. As such, we have a few requirements:

\begin{itemize}
\item
All materials must be written in \LaTeX\ and compiled into pdfs

\item
The documents must be free of grammar or spelling mistakes 

\item
Materials must be aptly named with a short and descriptive title

\item
The template.tex file provided in \underline{templates/} must be used as the starting point for every document

\item
The materials must be relevant to the topic and be as complete as possible (i.e. No half written explanations on methods or classes)

\item
Any .java files must adhere to the standards of .java files that exist in \underline{src/} 

\item
Do not include any `helper' files (*.out, *.aux, *.log, etc) in your commit, they will be ignored by default with \texttt{.gitignore}
\end{itemize}

\newpage
\section*{Style}

This section discusses the recommended practices with regard to special formatting of file paths, file names, commands, and URLs.

\texttt{TL;DR}

	\quad Use \underline{underline} to show location, use \texttt{typewriter} to emphasize content, and use \textbf{boldface} to represent a recurring concept.

\begin{itemize}
	\item Underline all descriptors of files and directories using '\textbackslash{}underline', including relative or absolute paths (e.g., \underline{~/git/projects/docs/README.txt} and \underline{myDirectory/})

	\item Add a forward slash ('/') to specify a directory. (The path \underline{path/to/location/} indicates that 'location' is a \textbf{directory}, where as \underline{path/to/location} indicates that 'location' is a \textbf{file})

	\item When referring to a file by its filename and not its path, use '\textbackslash{}texttt' to emphasize its name (e.g., \texttt{Counter.java} and \texttt{.gitignore})

	\item Use '\textbackslash{}texttt' to emphasize a particular terminal command or program (e.g., \texttt{sudo apt-get upgrade} and \texttt{python2.7})

	\item Use '\textbackslash{}url' to indicate the placement of a URL (e.g., \url{https://google.com/} and \url{git@github.com:TimMagoun/ProgrammingLesson.git}). This command is included in the \texttt{url} package

	\item Use '\textbackslash{}textbf' to emphasize a vocabulary or important concept in a lesson (e.g., \textbf{Variables} and \textbf{Inheritance})

	\item Use the package \texttt{minted} to format and display code in all documents
\end{itemize}

\section*{Environment Set-Up}
This section aims to briefly describe the steps to set up the environment required to compile
\texttt{template.tex}

\begin{enumerate}
\item
Install OS-specific version of \LaTeX\ (MacTex, MiKTeX, TexLive, etc)

\item
Install or Update \texttt{python2.7}

\item
Install \texttt{pip} if you don't already have it and make sure that it is at the most recent
version. \url{https://pip.pypa.io/en/stable/installing/}

\item
Install \texttt{pygments} by running: \texttt{pip install Pygments}

\item
Install minted from either the \LaTeX\ distro's package manager or install manually from here: 
\url{https://github.com/gpoore/minted}

\item
To compile \texttt{template.tex}, run \texttt{pdflatex -shell-escape template.tex} in \underline{template/}. This should create a \texttt{template.pdf} file in the same directory as \texttt{template.tex}, as well as many auxiliary files such as \texttt{template.log} and \texttt{template.aux}
\end{enumerate}


\section*{Organization}
Now that you're ready to compile \LaTeX, let's look at the structure of this repository.

\begin{itemize}
	\item \underline{bin/}, \underline{src/}, and \texttt{.project} are all part of an Eclipse project
	\item \texttt{.git} and \texttt{.gitignore} are essential parts of this repository 
	\item \underline{contributing/} is the directory above this instructional pdf and tex file. It serves to be an informational guide to those who wants to contribute to this project
	\item \underline{environment/} contains extensive guides for installing tools needed to program
	\item \underline{lessons/} contains documents that supplement this lesson. They could also be used as standalone resources for teams without prior programming experience
	\item \underline{resources/} contains helpful guides found on the internet that we thought would be helpful to those who are starting out
	\item \underline{problem\_bank/} contains a set of problems that may be used for a pre/final assessment. It also contains homework problem sets for each lesson, which could be reviewed in class for further discussion
	\item \underline{syllabus/} contains the syllabus for the lessons, and it includes the instruction to set up Eclipse
	\item \underline{template/} contains the template file for creating new documents
	\item \underline{tests/} contains the compiled tests

\end{itemize}



\end{document}
