\documentclass[11pt,fleqn]{article}

%% This first part is the document header, which you don't need to edit.
%% Scroll down to \begin{document}

\usepackage[latin1]{inputenc}
\usepackage{enumerate}
\usepackage[hang,flushmargin]{footmisc}
\usepackage{mdframed}
\usepackage{color}
\usepackage{minted}
\usepackage{datetime}
\usepackage{hyperref}

\newcommand{\mil}[2][java]{\mintinline{#1}|#2|}

\setlength{\oddsidemargin}{0px}
\setlength{\textwidth}{460px}
\setlength{\voffset}{-1.5cm}
\setlength{\textheight}{20cm}
\setlength{\parindent}{0px}
\setlength{\parskip}{10pt}
\hypersetup{
    colorlinks=true
}



\begin{document}
\title{Intro to WPILibJ}%Insert Title here
\author{Aravind Koneru}
\date{\textit{Compiled on} \today \hspace{1mm} at \currenttime}
\maketitle

\section{Introduction}

The WPI Library is a Java API that FIRST developed with the help of WPI (Worcester Polytechnic Institute)
to help students write readable code for the robot. Despite being very high level code, the WPI Library
allows users to have nearly full control over every aspect of the robot. The purpose of this document
is to introduce you to the big ideas from WPI Lib and explain important concepts required for programming
the robot. Rather than being a document detailing exactly what to type, this document outlines how the
WPI Lib actually works, what type of code goes where, and the coding styles/methodologies required by
the MidKnight Inventors. 

\section{Command Based Code}

There are various `ways' to write code for the robot. The standard for the MidKnight Inventors is to use
the \textbf{Command Based} coding style. The main reason that we choose to write our code in a command
based way is because it keeps the code modular, clean, and easy to debug. Also, command based code is
the most flexible and allows users to easily create additionally functionality that isn't already a part
of the WPI Lib. That being said, the command based style of writing code is the most conceptually complex
and requires a thorough understanding of what the code is doing at exactly every point. 

\section{The Master Classes}

When you first make a new command based robot program, there are three default classes present in the
\texttt{org.usfirst.teamXXXX.robot}. I refer to these three classes as the the \textbf{master classes}
since they directly control what the robot is doing. All the other code `plugs' into one of the three
master classes. 

\subsection{\texttt{Robot.java}}

\texttt{Robot.java} is both the simplest and most of the three master classes. \texttt{Robot.java} contains
instructions for what the robot should be doing during every state that it could be in. Since we are using
a command based system, all of our subsystems will be initialized in this class. You can think of the
entire robot being in a constant state machine controlled by the \textbf{Field Management System} (FMS).
\texttt{Robot.java} contains methods for all the different states that the robot could be in. 


Let's break down \texttt{Robot.java} to understand what it's doing at every point. Here's the code that
you would see in \texttt{Robot.java} if you made a new project: 

\begin{minted}{java}

package org.usfirst.frc.team1923.robot;
...

public class Robot extends IterativeRobot {

	public static final ExampleSubsystem exampleSubsystem = new ExampleSubsystem(); 
	public static OI oi;

    Command autonomousCommand;
    SendableChooser chooser;

    public void robotInit() {
		oi = new OI();
        chooser = new SendableChooser();
        chooser.addDefault("Default Auto", new ExampleCommand());
//        chooser.addObject("My Auto", new MyAutoCommand());
        SmartDashboard.putData("Auto mode", chooser);
    }
	
    public void disabledInit(){

    }
	
    public void disabledPeriodic() {
		Scheduler.getInstance().run();
    }


    public void autonomousInit() {
        autonomousCommand = (Command) chooser.getSelected();
        
		/* String autoSelected = SmartDashboard.getString("Auto Selector", "Default");
		switch(autoSelected) {
		case "My Auto":
			autonomousCommand = new MyAutoCommand();
			break;
		case "Default Auto":
		default:
			autonomousCommand = new ExampleCommand();
			break;
		} */
    	
    	// schedule the autonomous command (example)
        if (autonomousCommand != null) autonomousCommand.start();
    }

    public void autonomousPeriodic() {
        Scheduler.getInstance().run();
    }

    public void teleopInit() {
		
    }


    public void teleopPeriodic() {
        Scheduler.getInstance().run();
    }
    
    public void testPeriodic() {
        LiveWindow.run();
    }
}
\end{minted}

Let's start by breaking down each method to see what it does. 

\subsubsection{\mint{java}|public void robotInit()|}
When the robot is first turned on and enabled, it will run this set of instructions. \texttt{robotInit
()} is where you should declare all of your defaults and initialize and logging. The most important thing
to remember is that \texttt{robotInit()} contains anything that you want the robot to do every time it's
placed on the field. Let's say you wanted an intake to always start in the down position, then you would
call that code here in \texttt{robotInit()}.

\subsubsection{\mint{java}|public void disabledInit()|}
Whenever the robot is disabled, this method is called. Although it's not used as often as \texttt{robotInit
()}, \texttt{disabledInit()} can have some important use cases. Occasionally, when the robot is disabled,
a command may be in the middle of running or there might be an over-arching command that runs asynchronously
that needs to be stopped. Stopping such a command would be done in \texttt{disabledInit()}. Additionally,
when the robot is disabled there are sometimes changes that we want to make to the robot without actually
pressing any buttons, this can be in the code fairly easily. Examples of such functionality would be making
sure an arm is retracted all the way so that the drive team can easily move the robot from the field to
the cart. 

\subsubsection{\mint{java}|public void disabledPeriodic()|}
As the method name suggests, this code will be called periodically while the robot is disabled. This means
that the code present in this method will run every few hundred milliseconds. While this method is rarely
used by teams because it presents a safety issue, it can be used for novelty actions. If there are rgb
leds on the robot, they can be controlled even while the robot is disabled by using this method. Once
again however, this \textbf{method should not be used} without the \textbf{explicit permission}
from the programming co-captain. 

\subsubsection{\mint{java}|public void autonomousInit()|}
\texttt{autonomousInit()} is called at the beginning of \textbf{autonomous mode} (commonly referred to
as auton). This function receives user input to determine which auton to run, and then runs said auton.
It is important to make sure that this part of the code has fall-backs because if an auton choose or the
desired input method doesn't work, then you will be left dead on the field for auton (which usually turns
out to be a bad thing). As such, we require that there are fail safes in place to ensure that \textit
{an auton} will run even if it isn't the best one.

\subsubsection{\mint{java}|public void autonoumousPeriodic()|}
Similar to \texttt{disabledPeriodic()}, this method runs every hundred or so milliseconds during auton.
Unlike \texttt{disabledPeriodic()}, this method can be used in any way that you please since it doesn't
pose a safety problem.

\subsubsection{\mint{java}|public void teleopInit()|}
Whenever tele-operated (teleOp) mode is enabled, the FMS will call this method. You would use this method
to default to a position after auton and use \texttt{robotInit()} to default to a position before auton.
If there are senors that need to be zeroed or solenoids that need to be activated before the start of
teleOp but after auton, then this is where that code would. 

\subsubsection{\mint{java}|public void teleopPeriodic()|}
When teleOp mode is enabled, this code is called periodically (every 20ms or so). If you need want to
get back sensor readings, you would do those things in this method. Being able to see the status of current
actions such as the position of an arm or the current readout of encoders is very important to the drive
team since they won't always be able to see the robot from behind the glass. Also, running some sort of
logging is useful while debugging problems so that you can see what exactly is going wrong. 

\subsubsection{\mil{public void testPeriodic()}}
This method isn't really important and usually remains empty. On the driver's station, there is an option
for running the robot in a `test mode' which just removes the timer. We don't really use this functionality
of the driver's station on 1923 and we just tend to enable teleOp instead for testing purposes. 

\subsection{\texttt{OI.java}}

\texttt{OI.java} is where you would declare and initialize all of your controllers and map specific actions
of the robot to different functions. This class defaults to showing up empty and there are no special
functions that are called periodically like in \texttt{Robot.java}. Let's go through an example mapping.

\textbf{Disclaimer:} If you can't get a general idea of what the code is doing from reading it (ie.
creating a non-basic variable, constructors), you need to brush up on your basic java. The rest of
the document is heavily focused on code and doesn't have as many conceptual ideas as above. 

\begin{minted}{java}
package org.usfirst.frc.team1923.robot;

import edu.wpi.first.wpilibj.Joystick;
import edu.wpi.first.wpilibj.buttons.JoystickButton;

import org.usfirst.frc.team1923.robot.commands.*;
import org.usfirst.frc.team1923.robot.utils.*;

/**
 * This class is the glue that binds the controls on the physical operator
 * interface to the commands and command groups that allow control of the robot.
 */
public class OI {

    public Joystick leftStick, rightStick;
    public JoystickButton leftTrigger, rightTrigger;

    public OI() {
        //Point 1
        leftStick = new Joystick(1);
        rightStick = new Joystick(2);

        //Point 2
        leftTrigger = new JoystickButton(leftStick, 1);
        rightTrigger = new JoystickButton(rightStick, 1);
        
        //Point 3
        leftTrigger.whenPressed(new GearShiftCommand(true));
        rightTrigger.whenPressed(new GearShiftCommand(false));
    }
}
\end{minted}

\subsubsection{Point 1}
We declared two Joystick objects (\texttt{leftStick} and \texttt{rightStick}) earlier in \texttt{OI.java},
but we didn't initialize them. At this point in the code we are finally initializing our two \texttt{Joystick}
objects and assigning them to usb ports. 

\mil{leftStick = new Joystick(1);}

This means that we are assigning our `\textit{leftStick}' to usb port 1. This translates to mean that
the joystick that will control the left side of the robot (assuming that we are using a drive system known
as tank drive), must be plugged into port 1. Similarly, the joystick that will controlling the right side
of the robot must be plugged into port 2. It's always a good idea to have these mappings written down
somewhere on the driver's station or to test to make that the joysticks are plugged into their proper
port before each match/practice. 

\subsubsection{Point 2}
Once again, we are initializing two objects that we declared earlier in \texttt{OI.java}. This part of
the code is pretty self-explanatory so if you don't understand what the parameters are or what is being
initialized, please refer to the \href{http://first.wpi.edu/FRC/roborio/release/docs/java/}{WPILibJ Documentation
on \texttt{JoystickButtons}}.

\subsubsection{Point 3}
Until this point, we've just been initializing and declaring joysticks, controllers, and buttons in the
code. Point 3 is where the actual work is being done to translate pressing a button into causing the robot
to perform an action. 

\mil{leftTrigger.whenPressed(new GearShiftCommand(true));}

In english, this line is stating that when the leftTrigger button is pressed, call the GearShiftCommand.
In addition to \texttt{whenPressed}, there are a variety of options such as \texttt{whileHeld} or \texttt
{whenReleased} to accommodate all types of control schemes. Thinking of the big picture, lines of code
like this actually map actions to buttons. 

In a real robot, there are going to be many more commands and many more button mappings, but the general
idea remains the same so it is important to remember to name your commands and buttons in a manner that
allows you to quickly make changes if needed. 

\section{Commands}
In \texttt{RobotMap.java} we mapped button presses and joystick movements to commands using
functions like \texttt{.whenPressed}, but what really are commands? Simply put, commands are a
structured way of interfacing with subsystems. This means that commands themselves don't cause an
immediate physical change on the robot, rather, commands call methods in a manner that causes a
physical change. Let's look at the structure of a command:

\begin{minted}{java}

package org.usfirst.frc.team1923.robot.commands;

import org.usfirst.frc.team1923.robot.Robot;

import edu.wpi.first.wpilibj.command.Command;

/**
 *
 */
public class IntakePistonCommand extends Command {

    public IntakePistonCommand() {
        requires(Robot.intakePistonSubsystem);
    }

    // Called just before this Command runs the first time
    protected void initialize() {
        Robot.intakePistonSubsystem.toggle();
    }

    // Called repeatedly when this Command is scheduled to run
    protected void execute() {

    }

    // Make this return true when this Command no longer needs to run execute()
    protected boolean isFinished() {
        return true;
    }

    // Called once after isFinished returns true
    protected void end() {
        
    }

    // Called when another command which requires one or more of the same
    // subsystems is scheduled to run
    protected void interrupted() {
    }
}
\end{minted}

The first important thing to note is that all commands must extend the \texttt{Command} class that
is provided by WPILibJ. The \texttt{Command} class required the following methods (\textbf{Note:}
if you don't understand the method modifiers, please review the intro to java document):
\begin{itemize}
    \item
    \mil{public void initialize()}

    \item
    \mil{protected void execute()}

    \item
    \mil{protected boolean isFinished()}

    \item
    \mil{protected void end()}

    \item
    \mil{protected void interrupted()}
\end{itemize}

It is important to understand that Commands are called in the following loop:

\begin{minted}{java}

initialize();
while(!isFinished()){
    execute();
}
end();
\end{minted}

It is up to the programmer to understand the ramifications of the code running in this loop and how
to write robot code fully utilizes this system. 

\subsection{Constructor}
Notice that in the constructor, we use \texttt{requires} and pass in a subsystem object from
Robot.java. The \texttt{requires} command essentially imports a subsystem and lets the code know
that you will be calling methods from a specific subsystem. Failing to include this line of code
will result in the command flat out not working. 

\subsection{\mil{public void initialize()}}
This method is called immediately after the command every time the command is called. Let's say
that I want an encoder to reset before an arm moves, then you would put that encoder reset code in
the \texttt{initialize()}. A more likely use case is that you have a solenoid that you want to
toggle. If you want to just toggle a solenoid, that can easily be done in \texttt{initialize}
because it does not require any continuous action to work. Similarly, you could not put code for an
intake roller in \texttt{initialize} because it requires continuous action. 

\subsection{\mil{protected void execute()}}
This method is repeatedly called and should contain any code that causes a continuous action. For
example, you want an intake roller to constantly intake a ball while a button is pressed. In that
case, you would put that code in this method. 

\subsection{\mil{protected boolean isFinished()}}
Simply put, this method should return a boolean to indicate whether or not to stop running the
command. Ideally, this method would return a boolean based of off a sensor value or simply return
\texttt{true} or \texttt{false}. For the sake of readability, this method should also be as clear
and concise as possible to make it easier to trouble shoot. 

\subsection{\mil{protected void end()}}
This code is called at the end of method and effectively serves the same purpose as initialize.

\subsection{\mil{protected void interrupted()}}
Because of how the scheduler works, there needs to be an \texttt{interrupted} method. This method
handles the situation in which a command is in the middle of running, but the Scheduler needs to run
another command. In this case, the cleanest thing to do would to be to call \texttt{end()}, but this
method allows for another level of complexity and control over the robot's actions.

\section{Subsystems}
A subsystem is where the code to cause a physical change on the robot is written. Most often, a
subsystem will have a series of user-defined methods that only do a single thing and change the
status of objects declared in \texttt{RobotMap.java}. Here's an example of a subsystem: 

\begin{minted}{java}
package org.usfirst.frc.team1923.robot.subsystems;

import org.usfirst.frc.team1923.robot.Robot;
import org.usfirst.frc.team1923.robot.RobotMap;

import edu.wpi.first.wpilibj.DoubleSolenoid;
import edu.wpi.first.wpilibj.command.Subsystem;

/**
 *
 */
public class IntakePistonSubsystem extends Subsystem {
    
    // Put methods for controlling this subsystem
    // here. Call these from Commands.
    
    private boolean isDown;

    public void initDefaultCommand() {
        
    }
    
    public void toggle(){
        if(isDown){
            intakeUp();
        } else{
            intakeDown();
        }
    }
    
    public void intakeDown(){
        Robot.defensePistonSubsystem.up();
        RobotMap.intakeSolenoid.set(DoubleSolenoid.Value.kForward);
        isDown = true;
    }
    
    public void intakeUp(){
        RobotMap.intakeSolenoid.set(DoubleSolenoid.Value.kReverse);
        isDown = false;
    }
    
    public boolean intakePosition(){
        return isDown;
    }
}
\end{minted}

Most of the methods above are user-defined and change based on what function the subsystem needs to
perform. The only method that is inherited from the Subsystem class is \texttt{initDefaultCommand
()}. The best way to explain how this method works is by thinking about a real world example. Let's
say that you have some auto-aiming feature of your robot that requires the use of a drivetrain
subsystem. Ideally, you would also be able to control the robot from the joysticks immediately after
using the auto aim. In this case, you would put the following code in the \texttt{initDeafultMethod
()}:

\mil{setDefaultCommand(new MyDefaultCommand());}

Where \texttt{new MyDeafultCommand()} would be the command that correlates to driving with
joysticks. Essentially what this does is that once a command is done using a subsystem, if there is
a default command defined in the subsystem, then that default command will be called immediately
after. 

Other than this single method, the rest of the methods are user defined and you can have as many or
as few as you need/want. Ideally, you would keep the subsystem as concise as possible and have some
sort of enum value or boolean to indicate the status of the subsystem to display on the
SmartDashboard. 

\section{Review}
By this point, you should be able to...
\begin{itemize}
\item
Describe the role of each of the master classes
\item
Describe the loop system of the Commands
\item
Explain where everything needs to be defined (ex. a command needs to be declared here and
subsystems need to be declared here)
\item
Write simple commands and subsystem
\item
Know how to write code that maps commands to buttons
\item
\textbf{Create a UML Diagram detailing exactly how the Command-based programming approach works}
\end{itemize}

Please attempt to do all of these things, especially the last bullet, to ensure that you understand
the basics of writing robot code. 

\section{Required Coding Style}
TODO: Discuss with Tim

\end{document}
