\documentclass[11pt,fleqn]{article}

%% This first part is the document header, which you don't need to edit.
%% Scroll down to \begin{document}

\usepackage[latin1]{inputenc}
\usepackage{enumerate}
\usepackage[hang,flushmargin]{footmisc}
\usepackage{mdframed}
\usepackage{minted}
\usepackage{color}
\usepackage{datetime}
\usepackage{graphicx}
%%Please place all images used in documents in the images folder
\graphicspath{ {../images/} }

\setlength{\oddsidemargin}{0px}
\setlength{\textwidth}{460px}
\setlength{\voffset}{-1.5cm}
\setlength{\textheight}{20cm}
\setlength{\parindent}{0px}
\setlength{\parskip}{10pt}

\newcommand{\mil}[2][java]{\mintinline{#1}|#2|}
%% This command allows quick use of \mintinline feature, default language is java.

%% USAGE: \mil (optional)[<language>] {content}

%% EXAMPLE: \mil[python]{if not x == 3}
%% 			\mil{if (x.equals(y)}

\begin{document}
\title{Lesson 1 Problem Set}%Insert Title here
\author{Tim Magoun and Aravind Koneru}
\date{\textit{Compiled on} \today \hspace{1mm} at \currenttime}
\maketitle

\begin{center}
\textbf{Do these problems for additional practice and challenge.}
\end{center}

\begin{enumerate}
\item 
The modulus operator in Java is used to find the remainder of a division. This operator takes the
form of \% in Java with an example usage of \mil{int remainder = dividend } \% \mil{divisor}. For example,
13\%5 = 3. 

Use the knowledge learned in the first lesson, write the code in the given outline to simulate the modulus operator:
\inputminted{java}{../../src/ps_1/Modulus.java}

\item
\textbf{Challenge}:
Create a variable that will round off decimal values to the nearest integer value. For example, if I input
10.5, it should round up to 11. Similarly, if the input is 10.3, it should round down to 10. Please do
not use a built in java method to accomplish this task. 

\inputminted{java}{../../src/ps_1/Rounder.java}

\end{enumerate}

\end{document}
